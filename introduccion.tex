\begin{chapter}*{Introducción}
Un problema recurrente en diversas áreas del álgebra consiste en, dados un objeto algebraico $A$ y un subobjeto $B$, poder entender la estructura del cociente $A/B$. En su trabajo seminal \cite{Ber78}, Bergman introduce el \emph{lema del diamante}, una técnica para poder atacar este problema en el contexto de la teoría de $k$-álgebras. Esencialmente, uno puede entender a las relaciones dadas por los elementos del ideal $B$ como reglas de reescritura para elementos del cociente $A/B$. Si el sistema de reglas de reescritura verifica una cierta condición de confluencia, entonces somos capaces de realizar cómputo explícito en el cociente de una manera sumamente sencilla.

El lema del diamante de Bergman es un resultado muy versátil; en la misma publicación donde fue enunciado aparecen diversas versiones del mismo para múltiples estructuras algebraicas. Más recientemente, en \cite{SAV15} se describe un resultado análogo para completaciones de $k$-álgebras. En este trabajo empleamos esta variación para estudiar una familia particular de cocientes de este tipo de objetos.

En \cite{DWZ08} se describe un procedimiento que permite asignarle un \emph{quiver con potencial}, o \emph{QP}, a una triangulación de una superficie. Un quiver es un multigrafo dirigido, y un potencial es una combinación lineal de ciclos en la completación del álgebra de caminos asociada al quiver. El \emph{álgebra Jacobiana} asociada al QP es un cociente particular de dicha completación; explícitamente, es aquel que se obtiene al cocientar el ideal cerrado generado por las derivadas cíclicas del potencial. En trabajos como \cite{Lad12} y \cite{TVD12} se estudia el problema de determinar si las álgebras Jacobianas procedentes de triangulaciones de superficies cerradas son de dimensión finita. En dichas publicaciones se establece que esto es válido, con la única posible excepción de la esfera con 4 punciones. En esta tesis estudiamos este tipo de problemas, pero empleando como herramienta principal el lema del diamante. Más aún, introducimos un procedimiento análogo para generar un QP a partir de una subdivisión poligonal arbitraria de una superficie y abordamos este mismo tipo de problemas en esta nueva situación.

Nuestro trabajo se organiza de la siguiente manera. En el primer capítulo introducimos los preliminares necesarios para definir con precisión los QPs asociados a una triangulación de una superficie, y sus correspondientes álgebras Jacobianas. En el segundo capítulo introducimos el lema del diamante de Bergman en su versión para $k$-álgebras y luego en una variante para completaciones de álgebras de caminos, que será la que usaremos principalmente. Incluimos además varios ejemplos para ilustrar cómo se utiliza el lema del diamante en situaciones concretas.

El tercer capítulo contiene los resultados principales de la tesis. Comenzamos estudiando el caso de la esfera con 4 punciones, que era el único caso sin cubrir en \cite{Lad12}, y determinamos que el álgebra Jacobiana asociada es de dimensión infinita (más aún, calculamos su serie de Hilbert). Luego, estudiamos el problema de la finito-dimensionalidad en el caso de álgebras Jacobianas procedentes de subdivisiones arbitrarias. Construimos familias infinitas de dichas álgebras tanto de dimensión finita como infinita, mostrando que la situación en el caso poligonal es remarcablemente diferente al caso triangular. Finalmente, producimos sistemas de reescritura confluentes para 3 familias de subdivisiones poligonales de la esfera: las pirámides, los prismas y los antiprismas. La familia de pirámides provee además de una familia de contraejemplos a la generalización de un teorema de Ladkani (concerniente a la relación entre la dimensión del álgebra y la elección de escalares en el potencial) al caso poligonal.

En el último capítulo, utilizamos estos sistemas de reescritura para calcular invariantes de tipo cohomológico para sus álgebras Jacobianas asociadas. En particular, computamos el centro de dichas álgebras y probamos que admiten derivaciones no triviales.

Finalmente, incluimos un apéndice en el que anexamos y explicamos el funcionamiento de un programa escrito en \texttt{SageMath}, el cual desarrollamos para facilitar el cálculo de sistemas de reescritura.
\end{chapter}