\begin{chapter}{Invariants}
In the previous chapter, we produced confluent rewriting systems for several Jacobian algebras. As we have noted, such a rewriting system provides a basis for the algebra as a $k$-vector space, which is given by the irreducible monomials. In this chapter, we will make use of these bases to compute some invariants of the families of algebras considered previously.
\begin{section}{The center}
\textcolor{red}{TODO}
\end{section}
\begin{section}{Derivations}
Let $A$ be a $k$-algebra. A \emph{derivation} of $A$ is a $k$-linear morphism $f:A\to A$ satisfying the \emph{Leibniz rule}
\[f(ab)=f(a)b+af(b).\]
The set of derivations of $A$, which we will denote as $D(A)$, is a Lie algebra with Lie bracket given by the commutator
\[[f,g] = f\circ g - g\circ f.\]
Recall that, given a quiver $Q$, we denote its vertex span $k^{Q_0}$ as $R$. Since a Jacobian algebra $A$ is an $R$-bimodule, it makes sense to consider the Lie subalgebra of $R$-linear derivations of $A$, which we will denote $D_R(A)$. The following easy observations will greatly simplify our work:
\begin{obs}\label{derivation-quotient} Let $A$ be a $k$-algebra and $I\subseteq A$ an ideal. If $f$ is a derivation of $A$ such that $f(I)\subseteq I$, then the induced linear map $\hat{f}:A/I\to A/I$ is a derivation of $A/I$. Moreover, if $R$ is a set of generators for $I$, it suffices to see that $f(R)\subseteq I$ to show that $f(I)\subseteq I$.
\end{obs}
\begin{proof} Since $f(I)\subseteq I$, the ideal $I$ is contained in the kernel of $\pi\circ f:A\to A/I$, where $\pi$ denotes the natural projection. Thus there is a well defined linear map $\hat{f}:A/I\to A/I$ which obviously satisfies the Leibniz rule, since $f$ does.

As for the second assertion, suppose $x\in I$. Then $x=\sum_{i=1}^n a_ir_ib_i$, where $r_i\in R$ and $a_i, b_j\in A$. Therefore, the Leibniz rule implies that
\[f(x) = \sum_{i=1}^n f(a_ir_ib_i)=\sum_{i=1}^n \left(f(a_i)r_ib_i + a_if(r_i)b_i + a_ir_if(b_i)\right)\in I\]
since by hypothesis $f(R)\subseteq I$.
\end{proof}
\begin{obs}\label{derivation-vertices} Let $v\in R$ be a stationary path in a Jacobian algebra $A$. If $f$ is an $R$-linear derivation of $A$, then $f(v)=0$.
\end{obs}
\begin{proof} Using the Leibniz rule, the $R$-linearity of $f$ and the fact that $v=v^2$, we have that
\[f(v)= f(v^2)= f(v)v + vf(v)=f(v^2)+ f(v^2)=2f(v^2)=2f(v)\]
and so $f(v)=0$.
\end{proof}
\begin{obs}\label{endpoints} Let $B$ be a path basis for a Jacobian algebra $A$. If $f$ is an $R$-linear derivation of $A$ and $p\in B$, then $f(p)$ is a linear combination of elements of $B$ only involving paths sharing the same endpoints as $p$.
\end{obs}
\begin{proof} Since $B$ is a basis for $A$, we have that
\[f(p) = \sum_{q\in B} \lambda_q q\]
for some scalars $\lambda_q\in k$. Let $x, y\in R$ be the source and the target of $p$, respectively. Then
\[\sum_{q\in B} \lambda_q q=f(p)=f(ypx) = yf(p)x =\sum_{q\in B} \lambda_q yqx,\]
and so $\lambda_q=0$ if $q$ has different endpoints than $p$.
\end{proof}
From now on, we will refer to $R$-linear derivations plainly as derivations.
\begin{subsection}{Pyramids with an even-sided base}
Let $n$ be an even number greater than 3. In \hyperref[pyramids]{Section \ref*{pyramids}}, we proved that the Jacobian algebra $A$ associated with a pyramid having an $n$-sided base is finite-dimensional. Following the notation used in that section, a basis of irreducible monomials for the rewriting system we found is given by
\[B=\{e_v,a^k,b^k,cb^j, da^j\},\] 
where $v$ runs through the set of vertices $Q_0$, $1\leq k\leq 2n-3$ and $0\leq j\leq n-2$. Once again, we are abusing notation, since for instance $a$ denotes several different paths of length 1. For simplicity, we will only study derivations $f$ that assign the same value to all paths sharing the same name. Thus, we may speak of the value of $f(a)$ in an unambigous manner.

If $f$ is such a derivation of $A$, then by \hyperref[endpoints]{Observation \ref*{endpoints}} we have that
\begin{align*}
f(a) &= \alpha a + \hat\alpha a^{n+1}\\
f(b) &= \beta b + \hat\beta b^{n+1}\\
f(c) &= \gamma c \\
f(d) &= \delta d,
\end{align*}
where all greek letters are scalars in $k$. We recall that the following relations generate the Jacobian ideal $I$:
\begin{align}
a^{n+1}+cd\\
b^{n+1}+dc\\
bd+da\\
ac+cb
\end{align}
By relation (1) and the Leibniz rule, we have that
\begin{align*}-(\gamma+\delta)cd&= -f(c)d - cf(d)\\
&= f(-cd)\\
&=f(a^{n+1})\\
&=\sum_{m=0}^{n} a^mf(a)a^{n-m}\\
&=\sum_{m=0}^{n} a^m(\alpha a + \hat\alpha a^{n+1}) a^{n-m}\\
&=(n+1)(\alpha a^{n+1} + \hat\alpha a^{2n+2})\\
&=(n+1)\alpha a^{n+1}\\
&=-(n+1)\alpha cd,
\end{align*}
and so $\gamma+\delta = (n+1)\alpha$. Reasoning analogously using relation (2), we conclude that $\gamma+\delta = (n+1)\beta$, and thus $\alpha=\beta$.

Relation (3) implies
\begin{align*}-(\beta+\delta)bd &=-(\beta+\delta)bd - \hat\beta bd^{n+1}\\
&=-f(b)d - bf(d)\\
&=f(-bd)\\
&=f(da)\\
&=f(d)a+df(a)\\
&=(\delta+\alpha)da + \hat\alpha da^{n+1}\\
&=(\delta+\alpha)da\\
&=-(\delta+\alpha)bd,
\end{align*}
and thus we get the equation $\alpha=\beta$ again. Relation (4) implies the same identity as well.

Now, if $g$ is a derivation of $\kQ$ satisfying equations \textcolor{red}{incluir link a $g(a)=\alpha a + \hat\alpha a^{n+1}...$}, its values on $a,b,c$ and $d$ completely determine it, since any path is either a stationary map (which is mapped to zero by \hyperref[derivation-vertices]{Observation \ref*{derivation-vertices}}) or equal to a unique product of $a,b,c$ and $d$, and thus its image is uniquely determined by the Leibniz rule. Moreover, as we have just seen, if $\alpha=\beta$ and $\gamma+\delta=(n+1)\alpha$ then the image of a set of generators of $I$ by $g$ is contained in $I$, and so $g$ induces a derivation of the Jacobian algebra \note{decir por qué esta bien hacerlo desde el álgebra sin completar} by \hyperref[derivation-quotient]{Observation \ref*{derivation-quotient}}.

We now write the values of a generic derivation of the form we described in terms of the basis $B$:
\begin{align*}
g(e_v) &= 0\\
g(a^k) &=\sum_{m=0}^{k-1} a^mg(a)a^{k-1-m}=k(\alpha a^k + \hat\alpha a^{n+k}) \\
g(b^k) &=\sum_{m=0}^{k-1}b^mg(b)b^{k-1-m}=k(\alpha b^k + \hat\beta b^{n+k})\\
g(cb^j) &= g(c)b^j + cg(b^j) = (\gamma+j\alpha) cb^j \\
g(da^j) &= g(d)a^j + dg(a^j) = (\delta+j\alpha) da^j = \left((n+j+1)\alpha-\gamma\right)da^j
\end{align*}
It is clear that these derivations \textcolor{red}{make up} a vector space $V$ of dimension 4. A basis for this space is given by $\{\alpha^*, \gamma^*, \hat\alpha^*, \hat\beta^*\}$, where $\alpha^*$ stands for the map defined by setting $\alpha=1$ and $\hat\alpha=\beta=\gamma=0$, and the other maps are defined analogously. We now write down the image of the basis $B$ in terms of these maps to ease computation:
\begin{center}
\begin{tabular}{ c | c | c | c | c }
	& 	$\alpha^*$ 		& $\gamma^*$	& $\hat\alpha^*$	& $\hat\beta^*$	\\
\hline
$e_v$ & 	0 			& 0 			& 0			& 0 \\
$a^k$ & 	$ka^k$ 		& 0			& $ka^{n+k}$	& 0 \\
$b^k$ & 	$kb^k$ 		& 0			& 0			& $kb^{n+k}$ \\
$cb^j$ & 	$jcb^j$ 		& $cb^j$		& 0			& 0 \\
$da^j$ & 	$(n+j+1)da^j$	& $-da^j$		& 0			& 0 
\end{tabular}
\end{center}

From this table we see that $B$ is a basis of eigenvectors for both $\alpha^*$ and $\gamma^*$, and thus these two maps commute. Moreover, $\langle a^k\rangle \oplus \langle b^k\rangle \oplus \langle e_v, cb^j, da^j\rangle$ is a decomposition of $A$ into both $\gamma^*$, $\hat\alpha^*$ and $\hat\beta^*$-invariant subspaces, and so these three maps commute with each other as well. These facts imply the vanishing of the brackets $[\alpha^*,\gamma^*]$,  $[\gamma^*, \hat\alpha^*]$, $[\gamma^*, \hat\beta^*]$ and $[\hat\alpha^*, \hat\beta^*]$. By direct computation \note{explicitar la cuenta?} using the table we see that $[\alpha^*,\hat\alpha^*]=n\hat\alpha^*$ and $[\alpha^*,\hat\beta^*]=n\hat\beta^*$. Therefore, $V$ is actually closed under the Lie bracket and so is a 4-dimensional Lie subalgebra of the algebra of derivations of $A$.

In fact, we may further characterize the Lie structure on $V$. Given Lie algebras $L_1$ and $L_2$ and an action by derivations $\cdot$ of $L_1$ on $L_2$, the \emph{semi-direct product $L_1\ltimes L_2$} is the $k$-vector space $L_1 \oplus L_2$ with Lie bracket given by
\[[(x_1,x_2), (y_1,y_2)] = ([x_1,y_1], [x_2,y_2]+x_1\cdot y_2 - y_1\cdot x_2).\]
It is now easy to see that $V$ is a semi-direct product of the abelian Lie algebras $\langle \alpha^*\rangle$ and $\langle \gamma^*, \hat\alpha^*, \hat\beta^*\rangle$, where $\alpha^*$ acts trivially over $\gamma^*$ and by multiplication by $n$ over $\hat\alpha^*$ and $\hat\beta^*$.
\end{subsection}
\end{section}
\end{chapter}