\begin{chapter}{Cohomological properties}
In the previous chapter, we produced confluent rewriting systems for several Jacobian algebras. As we have noted, such a rewriting system provides a basis for the algebra as a $k$-vector space, which is given by the irreducible monomials. In this chapter, we will make use of these bases to compute some invariants of the families of algebras considered previously. These invariants are closely related to the first two Hochschild cohomology modules of those algebras, which we will now introduce.

Given an associative $k$-algebra $A$, we define its \emph{enveloping algebra} as $A^e=A\otimes_k A^{\mathrm{opp}}$. The product on $A^e$ is given by $(a\otimes b)(c\otimes d)=ac\otimes db$.
There is a natural equivalence between the category of $A$-$A$-bimodules and the category of left $A^e$-modules, which we will consider an identification. Therefore, if $M$ is an $A$-$A$-bimodule, it makes sense to compute $\Ext^\bullet_{A^e}(A,M)$, which we will call the \emph{Hochschild cohomology of $A$ with coefficients in $M$} and denote $H^\bullet(A,M)$.
If $M=A$, this module will be plainly called the \emph{Hochschild cohomology of $A$}, and will be denoted as $HH^\bullet(A)$.

\begin{section}{The center}
Let $A$ be an associative $k$-algebra. The \emph{center} of $A$, denoted as $\mathcal{Z}(A)$, is the subset of all elements $x\in A$ such that $xa=ax$ for all $a\in A$, and is in fact a subalgebra of $A$. An algebra is called \emph{central} if $\mathcal{Z}(A)=k$. For example, the matrix algebra $M_n(k)$ is a central $k$-algebra.

\begin{lemma} Let $A$ be an associative $k$-algebra. Then $HH^0(A) = \mathcal{Z}(A)$.
\end{lemma}
\begin{proof} See \cite{Red01}.
\end{proof}

We will now compute the center of the Jacobian algebras associated to pyramids, prisms and antiprisms. In order todo so, we first make a useful observation:

\begin{obs} Let $x$ be a central element in a Jacobian algebra $A$. Then $x$ is a linear combination of \emph{loops}, that is, paths with identical source and target.
\end{obs}
\begin{proof} First of all, we stress that two paths which are equivalent in the Jacobian algebra must share the same endpoints, since the Jacobian ideal is generated by sums of paths satisfying this property. Therefore, it makes sense to speak of endpoints and loops in $A$.

Suppose now that $x$ is central in $A$. We write $x$ as a combination of linearly independent classes of paths $\sum_{i=1}^n \lambda_i p_i$. Let $v_j$ be the stationary path corresponding to the source of $p_j$. Since $x$ is central, we have that
\[\sum_{i=1}^n \lambda_i p_iv_j=(\sum_{i=1}^n \lambda_i p_i)v_j = xv_j=v_jx = v_j(\sum_{i=1}^n \lambda_i p_i)=\sum_{i=1}^n \lambda_i v_jp_i.\]
The scalar corresponding to $p_j$ in the left hand side of the equation is $\lambda_j$, since $p_jv_j=p_j$. However, the scalar corresponding to $p_j$ on the right hand side is $\lambda_j$ if $p_j$ has the vertex at $v_j$ as target, and 0 otherwise. Therefore, if $p_j$ is not a loop, then $\lambda_j=0$, and so $x$ is a linear combination of loops, as we wanted.
\end{proof}

As a consequence of the diamond lemma, we know that given a confluent rewriting system for a Jacobian algebra, the set of irreducible monomials forms a basis for it. This will be an essential tool to compute its center. In what follows, we will maintain the notation for paths used in the previous chapter.

\begin{prop} Let $A$ be the Jacobian algebra associated to a pyramid with $n$-sided base, where $n>3$. Then, if $n$ is even, $A$ is central. Otherwise, $A$ has non-trivial center, generated by \textcolor{red}{blah}.
\end{prop}
\begin{proof} Suppose first that $n$ is even. The confluent rewriting system found in the previous chapter induces the basis of irreducible paths 
\[B=\{e_v,a^k,b^k,cb^j, da^j\},\] 
where $v$ runs through the set of vertices $Q_0$, $1\leq k\leq 2n-3$ and $0\leq j\leq n-2$.\note{modificar el texto de las derivaciones para que no quede tan redundante} Since a central element must be 
\end{proof}
\end{section}

\begin{section}{Derivations}
Let $A$ be a $k$-algebra. A \emph{derivation} of $A$ is a $k$-linear morphism $f:A\to A$ satisfying the \emph{Leibniz rule}
\[f(ab)=f(a)b+af(b).\]
The set of derivations of $A$, which we will denote as $D(A)$, is a Lie algebra with Lie bracket given by the commutator
\[[f,g] = f\circ g - g\circ f.\]
If $x\in A$, the map $f_x(y) = xy-yx$ is a derivation. The set of such maps, which we call \emph{inner derivations}, is denoted $\Inn(A)$, and it is actually a $k$-subspace of $D(A)$. Our main interest in derivations relies on the following fact:

\begin{lemma} Let $A$ be an associative $k$-algebra. Then $HH^1(A) = D(A)/\Inn(A)$.
\end{lemma}
\begin{proof} See \cite{Red01}.
\end{proof}

Recall that, given a quiver $Q$, we denote its vertex span $k^{Q_0}$ as $R$. Since a Jacobian algebra $A$ is an $R$-bimodule, it makes sense to consider the Lie subalgebra of $R$-linear derivations of $A$, which we will denote $D_R(A)$. The following easy observations will greatly simplify our work:
\begin{obs}\label{derivation-quotient} Let $A$ be a $k$-algebra and $I\subseteq A$ an ideal. If $f$ is a derivation of $A$ such that $f(I)\subseteq I$, then the induced linear map $\hat{f}:A/I\to A/I$ is a derivation of $A/I$. Moreover, if $R$ is a set of generators for $I$, it suffices to see that $f(R)\subseteq I$ to show that $f(I)\subseteq I$.
\end{obs}
\begin{proof} Since $f(I)\subseteq I$, the ideal $I$ is contained in the kernel of $\pi\circ f:A\to A/I$, where $\pi$ denotes the natural projection. Thus there is a well defined linear map $\hat{f}:A/I\to A/I$ which obviously satisfies the Leibniz rule, since $f$ does.

As for the second assertion, suppose $x\in I$. Then $x=\sum_{i=1}^n a_ir_ib_i$, where $r_i\in R$ and $a_i, b_j\in A$. Therefore, the Leibniz rule implies that
\[f(x) = \sum_{i=1}^n f(a_ir_ib_i)=\sum_{i=1}^n \left(f(a_i)r_ib_i + a_if(r_i)b_i + a_ir_if(b_i)\right)\in I\]
since by hypothesis $f(R)\subseteq I$.
\end{proof}
\begin{obs}\label{derivation-vertices} Let $v\in R$ be a stationary path in a Jacobian algebra $A$. If $f$ is an $R$-linear derivation of $A$, then $f(v)=0$.
\end{obs}
\begin{proof} Using the Leibniz rule, the $R$-linearity of $f$ and the fact that $v=v^2$, we have that
\[f(v)= f(v^2)= f(v)v + vf(v)=f(v^2)+ f(v^2)=2f(v^2)=2f(v)\]
and so $f(v)=0$.
\end{proof}
\begin{obs}\label{endpoints} Let $B$ be a path basis for a Jacobian algebra $A$. If $f$ is an $R$-linear derivation of $A$ and $p\in B$, then $f(p)$ is a linear combination of elements of $B$ only involving paths sharing the same endpoints as $p$.
\end{obs}
\begin{proof} Since $B$ is a basis for $A$, we have that
\[f(p) = \sum_{q\in B} \lambda_q q\]
for some scalars $\lambda_q\in k$. Let $x, y\in R$ be the source and the target of $p$, respectively. Then
\[\sum_{q\in B} \lambda_q q=f(p)=f(ypx) = yf(p)x =\sum_{q\in B} \lambda_q yqx,\]
and so $\lambda_q=0$ if $q$ has different endpoints than $p$.
\end{proof}
From now on, we will refer to $R$-linear derivations plainly as derivations. We will also maintain the labeling of arrows presented in the previous chapter for the quivers associated to each polygonal subdivision.

\begin{subsection}{Pyramids with an even-sided base}
Let $n$ be an even number greater than 3. In \hyperref[pyramids]{Section \ref*{pyramids}}, we proved that the Jacobian algebra $A$ associated with a pyramid having an $n$-sided base is finite-dimensional. Following the notation used in that section, a basis of irreducible monomials for the rewriting system we found is given by
\[B=\{e_v,a^k,b^k,cb^j, da^j\},\] 
where $v$ runs through the set of vertices $Q_0$, $1\leq k\leq 2n-3$ and $0\leq j\leq n-2$. Once again, we are abusing notation, since for instance $a$ denotes several different paths of length 1. For simplicity, we will only study derivations $f$ that assign the same value to all paths sharing the same name. Thus, we may speak of the value of $f(a)$ in an unambigous manner.

If $f$ is such a derivation of $A$, then by \hyperref[endpoints]{Observation \ref*{endpoints}} we have that
\begin{equation}
\begin{aligned}
\label{pyramid-equations}
f(a) &= \alpha a + \hat\alpha a^{n+1}\\
f(b) &= \beta b + \hat\beta b^{n+1}\\
f(c) &= \gamma c \\
f(d) &= \delta d,
\end{aligned}
\end{equation}
where all greek letters are scalars in $k$. We recall that the following relations generate the Jacobian ideal $I$:
\begin{align}
a^{n+1}+cd \label{eq-a}\\
b^{n+1}+dc \label{eq-b}\\
bd+da \label{eq-c}\\
ac+cb \label{eq-d}
\end{align}
By relation \eqref{eq-a} and the Leibniz rule, we have that
\begin{align*}-(\gamma+\delta)cd&= -f(c)d - cf(d)\\
&= f(-cd)\\
&=f(a^{n+1})\\
&=\sum_{m=0}^{n} a^mf(a)a^{n-m}\\
&=\sum_{m=0}^{n} a^m(\alpha a + \hat\alpha a^{n+1}) a^{n-m}\\
&=(n+1)(\alpha a^{n+1} + \hat\alpha a^{2n+2})\\
&=(n+1)\alpha a^{n+1}\\
&=-(n+1)\alpha cd,
\end{align*}
and so $\gamma+\delta = (n+1)\alpha$. Reasoning analogously using relation \eqref{eq-b}, we conclude that $\gamma+\delta = (n+1)\beta$, and thus $\alpha=\beta$.

Relation \eqref{eq-c} implies
\begin{align*}-(\beta+\delta)bd &=-(\beta+\delta)bd - \hat\beta bd^{n+1}\\
&=-f(b)d - bf(d)\\
&=f(-bd)\\
&=f(da)\\
&=f(d)a+df(a)\\
&=(\delta+\alpha)da + \hat\alpha da^{n+1}\\
&=(\delta+\alpha)da\\
&=-(\delta+\alpha)bd,
\end{align*}
and thus we get the equation $\alpha=\beta$ again. Relation \eqref{eq-d} implies the same identity as well.

Now, if $g$ is a derivation of $\kQ$ satisfying the set of equations \eqref{pyramid-equations}, its values on $a,b,c$ and $d$ completely determine it, since any path is either a stationary path (which is mapped to zero by \hyperref[derivation-vertices]{Observation \ref*{derivation-vertices}}) or equal to a unique product of $a,b,c$ and $d$, and thus its image is uniquely determined by the Leibniz rule. Moreover, as we have just seen, if $\alpha=\beta$ and $\gamma+\delta=(n+1)\alpha$ then the image of a set of generators of $I$ by $g$ is contained in $I$, and so $g$ induces a derivation of the Jacobian algebra \note{decir por qué esta bien hacerlo desde el álgebra sin completar} by \hyperref[derivation-quotient]{Observation \ref*{derivation-quotient}}.

We now write the values of a generic derivation of the form we described in terms of the basis $B$:
\begin{align*}
g(e_v) &= 0\\
g(a^k) &=\sum_{m=0}^{k-1} a^mg(a)a^{k-1-m}=k(\alpha a^k + \hat\alpha a^{n+k}) \\
g(b^k) &=\sum_{m=0}^{k-1}b^mg(b)b^{k-1-m}=k(\alpha b^k + \hat\beta b^{n+k})\\
g(cb^j) &= g(c)b^j + cg(b^j) = (\gamma+j\alpha) cb^j \\
g(da^j) &= g(d)a^j + dg(a^j) = (\delta+j\alpha) da^j = \left((n+j+1)\alpha-\gamma\right)da^j
\end{align*}
It is clear that these derivations make up a vector space $V$ of dimension 4. A basis for this space is given by $\{\alpha^*, \gamma^*, \hat\alpha^*, \hat\beta^*\}$, where $\alpha^*$ stands for the map defined by setting $\alpha=1$ and $\hat\alpha=\beta=\gamma=0$, and the other maps are defined analogously. We now write down the image of the basis $B$ in terms of these maps to ease computation:
\begin{center}
\begin{tabular}{ c | c | c | c | c }
	& 	$\alpha^*$ 		& $\gamma^*$	& $\hat\alpha^*$	& $\hat\beta^*$	\\
\hline
$e_v$ & 	0 			& 0 			& 0			& 0 \\
$a^k$ & 	$ka^k$ 		& 0			& $ka^{n+k}$	& 0 \\
$b^k$ & 	$kb^k$ 		& 0			& 0			& $kb^{n+k}$ \\
$cb^j$ & 	$jcb^j$ 		& $cb^j$		& 0			& 0 \\
$da^j$ & 	$(n+j+1)da^j$	& $-da^j$		& 0			& 0 
\end{tabular}
\end{center}

From this table we see that $B$ is a basis of eigenvectors for both $\alpha^*$ and $\gamma^*$, and thus these two maps commute. Moreover, $\gamma^*$, $\hat\alpha^*$ and $\hat\beta^*$ commute with each other as well, since they act trivially outside of $\langle cb^j, da^j\rangle$, $\langle a^k\rangle$ and $\langle b^k\rangle$ respectively, and these three spaces are in direct sum. These facts imply the vanishing of the brackets $[\alpha^*,\gamma^*]$,  $[\gamma^*, \hat\alpha^*]$, $[\gamma^*, \hat\beta^*]$ and $[\hat\alpha^*, \hat\beta^*]$. By direct computation \note{explicitar la cuenta?} using the table we see that $[\alpha^*,\hat\alpha^*]=n\hat\alpha^*$ and $[\alpha^*,\hat\beta^*]=n\hat\beta^*$. Therefore, $V$ is actually closed under the Lie bracket and so is a 4-dimensional Lie subalgebra of the algebra of derivations of $A$.

In fact, we may further characterize the Lie structure on $V$. Given Lie algebras $L_1$ and $L_2$ and an action by derivations $\cdot$ of $L_1$ on $L_2$, the \emph{semi-direct product $L_1\ltimes L_2$} is the $k$-vector space $L_1 \oplus L_2$ with Lie bracket given by
\[[(x_1,x_2), (y_1,y_2)] = ([x_1,y_1], [x_2,y_2]+x_1\cdot y_2 - y_1\cdot x_2).\]
It is now easy to see that $V$ is a semi-direct product of the abelian Lie algebras $\langle \alpha^*\rangle$ and $\langle \gamma^*, \hat\alpha^*, \hat\beta^*\rangle$, where $\alpha^*$ acts trivially over $\gamma^*$ and by multiplication by $n$ over $\hat\alpha^*$ and $\hat\beta^*$.
\end{subsection}

Following the exact same modus operandi, we will now produce non-trivial derivations for the Jacobian algebras arising from prisms and antiprisms. Since in these cases there are more relations generating the Jacobian ideal than in the pyramidal case, the derivations will have to satisfy more constraints.

\begin{subsection}{Prisms}
As in the previous subsection, any derivation $\psi$ of the Jacobian algebra $A$ associated to a prism with an $n$-sided base must send paths to paths sharing the same endpoints. Therefore, by writing the image of the paths of length 1 by $\psi$ in the basis of irreducible monomials, we find that $\psi$ must satisfy:
\begin{align*}
\psi(a) &= \alpha a\\
\psi(b) &= \beta b\\
\psi(c) &= \gamma c \\
\psi(d) &= \delta d \\
\psi(e) &= \varepsilon e \\
\psi(f) &= \zeta f
\end{align*}
Notice that in this case there are no pairs of different paths sharing the same endpoints, which will make the rest of the computation quite easier. Mimicking the process carried out in the pyramidal case, we obtain constraints (in the right column) for the greek scalars using the relations that span the Jacobian ideal (in the left column) and the Leibniz rule:
\begin{align*}
de+a^{n-1} &  	&\delta+\varepsilon 	&=(n-1)\alpha\\
fc+b^{n-1}  &  	&\zeta+\gamma 		&=(n-1)\beta\\
ad+dcf	&	&\alpha+\delta		&=\delta+\gamma+\zeta\\
bf+fed	&	&\beta+\zeta		&=\zeta+\varepsilon+\delta\\
ea+cfe	&	&\varepsilon+\alpha	&=\gamma+\zeta+\varepsilon\\
cb+edc	&	&\gamma+\beta		&=\varepsilon+\delta+\gamma
\end{align*} 
Solving the linear system of equations on the right column, we find that $\alpha=\beta=0$, $\delta=-\varepsilon$ and $\gamma=-\zeta$. Once again, any set of scalars satisfying these constraints induces a derivation of the path algebra that passes to the quotient and induces a bonafide derivation of $A$. We thus obtain two linearly independent derivations of $A$, $\gamma^*$ and $\delta^*$, which are defined on paths of length 1 as
\begin{center}
\begin{tabular}{ c | c | c }
	& 	$\gamma^*$ 	& $\delta^*$\\
\hline
$a$ & 0 		& 0		\\
$b$ & 	0 		& 0		 \\
$c$ & 	$c$ 		& 0	 \\
$d$ & 	0		& $d$	\\
$e$ & 0		& $-e$ \\
$f$ & $-f$		& 0
\end{tabular}
\end{center}
\end{subsection}
\begin{subsection}{Antiprisms}
We follow the usual drill, this time for antiprisms. We start by writing down the image of the paths of length 1 by an eventual derivation $\psi$ in the basis of irreducible monomials, and find that $\psi$ must satisfy:
\begin{align*}
\psi(a) &= \alpha a +\hat\alpha a^{n+1}\\
\psi(b) &= \beta b + \hat\beta b^{n+1}\\
\psi(c) &= \gamma c \\
\psi(d) &= \delta d \\
\psi(e) &= \varepsilon e \\
\psi(f) &= \zeta f \\
\psi(g) &= \eta g \\
\psi(h) &= \theta h
\end{align*}
Once again, we have two pairs of paths sharing the same endpoints. We now find the constraints the scalars must verify using the relations in the Jacobian ideal:
\begin{align*}
feg+a^{n-1} &  	&\zeta+\varepsilon+\gamma 	&=(n-1)\alpha\\
chd+b^{n-1}  &  	&\gamma+\theta+\delta 		&=(n-1)\beta\\
ed+hbd	&	&\varepsilon+\delta		&=\theta+\beta+\delta\\
ce+bch	&	&\gamma+\varepsilon		&=\beta+\gamma+\theta\\
dc+gaf	&	&\delta+\gamma			&=\eta+\alpha+\zeta\\
hg+ega	&	&\theta+\eta			&=\varepsilon+\eta+\alpha\\
fh+afe	&	&\zeta+\theta			&=\alpha+\zeta+\varepsilon\\
gf+dbc	&	&\eta+\zeta				&=\delta+\beta+\gamma
\end{align*} 
The linear system of equations on the right column imposes the following constraints:
\begin{align*}
\alpha=\beta=0\\
\delta=-\gamma-\varepsilon\\
\eta=-\varepsilon-\zeta\\
\theta=-\varepsilon
\end{align*}
Any solution for this system of equations induces a derivation of the Jacobian algebra of the antiprism in the usual manner. We thus obtain five linearly independent derivations: $\hat\alpha^*$, $\hat\beta^*$, $\gamma^*$, $\varepsilon^*$ and $\zeta^*$, which values on the set of paths of length 1 we present on the following table:
\begin{center}
\begin{tabular}{ c | c | c | c | c | c }
	& $\hat\alpha^*$ 		& $\hat\beta^*$	& $\gamma^*$	& $\varepsilon^*$	& $\zeta^*$\\
\hline
$a$ & 	$a^{n+1}$ 			& 0			& 0			& 0 			& 0 \\
$b$ & 	0 				& $b^{n+1}$	& 0			& 0 			& 0 \\
$c$ & 	0 				& 0			& $c$			& 0 			& 0 \\
$d$ & 	0				& 0			& $-d$		& $-d$ 		& 0 \\ 
$e$ & 	0				& 0			& 0			& $e$ 			& 0 \\
$f$ & 	0				& 0			& 0			& 0 			& $f$ \\
$g$ & 	0				& 0			& 0			& $-g$		& $-f$ \\
$h$ & 	0				& 0			& 0			& $-h$		& 0 
\end{tabular}
\end{center}
\end{subsection}
\end{section}
\end{chapter}