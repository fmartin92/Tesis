\begin{chapter}*{Introduction}
A recurring problem in various branches of algebra consists in, given an algebraic object $A$ and a subobject $B$, understanding the structure of the quotient $A/B$. In his seminal work \cite{Ber78}, Bergman introduces the \emph{diamond lemma}, a tool used to tackle this problem in the context of $k$-algebras. Essentially, one may think of the relations given by elements in the ideal $B$ as rewriting rules for the elements in the quotient $A/B$. If the set of rewriting rules verifies a certain confluence condition, then we are able to carry out explicit computations in the quotient in a simple fashion.

Bergman's diamond lemma is a very versatile result; there are several versions of it for various algebraic structures on the same paper where it was first stated. More recently, in \cite{SAV15}, an analogous result for completions of $k$-algebras is described. In this work, we make use of this variation to study a particular family of quotients of this kind of objects.

In \cite{DWZ08}, the authors describe a procedure that assigns a \emph{quiver with potential}, or \emph{QP} for short, to a triangulation of a surface. A quiver is a directed multigraph, and a potential is a linear combination of cycles in the completion of the path algebra associated to said quiver. The \emph{Jacobian algebra} associated to the QP is a particular quotient of said completion; explicitly, it is the one obtained after modding out the closed ideal spanned by the set of cyclic derivatives of the potential. In works such as \cite{Lad12} and \cite{TVD12}, the authors study the problem of determining if Jacobian algebras arising from triangulations of closed surfaces are finite-dimensional. The validity of this result is established on these papers, with the possible exception of the case of the sphere with four punctures. In this thesis we study this kind of problems, but using the diamond lemma as our main tool. Moreover, we introduce a procedure that assigns a QP to an arbitrary polygonal subdivision of a surface, and we tackle the same kind of problems in this new setting.

Our work is organized as follows. In the first chapter we introduce the necessary preliminaries in order to define the QP associated to a triangulation of a surface, and its corresponding Jacobian algebra. In the second chapter we introduce Bergman's diamond lemma, first in the classical $k$-algebra setting and then we present a variation of it that applies over completions of path algebras, which we will use the most. We also include several examples, with the purpose of showing how the diamond lemma is used in concrete situations.

The third chapter contains our main results. We start off by studying the case of the sphere with four punctures, which is the only case not discussed in \cite{Lad12}, and we prove that the associated Jacobian algebra is infinite-dimensional (moreover, we compute its Hilbert series). We then study the finite-dimensionality problem in the case of Jacobian algebras arising from arbitrary polygonal subdvisions. We construct infinite families of said algebras, of both finite and infinite dimension, showing that the situation in the polygonal case is remarkably different to the triangular case. Finally, we produce confluent rewriting systems for three families of polygonal subdivisions of the sphere: pyramids, prisms and antiprisms. Pyramids also provide a family of counterexamples to the generalization of a theorem of Ladkani (concerning the relation between the dimension of the Jacobian algebra and the choice of scalars appearing in the potential) to the polygonal case.

In the last chapter, we use these rewriting systems to compute cohomological invariants of the associated Jacobian algebras. In particular, we compute the center of said algebras and we prove that they admit non-trivial derivations.

Finally, we include an appendix on which we present a program written in \texttt{SageMath}, which we developed to ease computations related to rewriting systems.
\end{chapter}